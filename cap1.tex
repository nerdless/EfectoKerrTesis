La Detecci\'on \'Optica por Efecto Kerr \'Optico (OKE)\cite{steve} es una t\'ecnica relativamente simple, no destructiva y no invasiva con excelente relaci\'on se\~nal-ruido y alta resoluci\'on temporal aplicada en muestras en fase l\'iquida para el estudio de la din\'amica molecular. El propósito de esta t\'esis es la implementación de dicha técnica para el estudio de la din\'amica molecular de CS$_2$.\\
\textcolor{red}{Aqu\'i no se si hablar de manera resumida de la t\'ecnica decir algo como}\\
\textcolor{blue}{En \'esta un pulso \'optico linealmente polarizado interacciona con la muestra que presenta un estado isotr\'opico inicial para posteriormente despu\'es de la perturbaci\'on induce una birrefringencia transitoria. Un segundo pulso, retrasado en tiempo incidiendo en el mismo punto en la muestra, con su plano de polarizaci\'on orientado a cierto \'angulo respecto al primero, experimentar\'a un cambio de polarizaci\'on debido a la birrefringencia inducida en el medio y emerger\'a con cierto grado de polarizaci\'on el\'iptica. Con un polarizador cruzado con el plano de polarizaci\'on del segundo pulso es colocado frente a un detector. La birrefringencia transitoria inducida en la muestra provocar\'a que cierta fracci\'on de la prueba llegue el detector. Esto es lo que se conoce como se\~nal OKE resuelta en tiempo ya que la relajaci\'on de la birrefringencia es monitoreada en del tiempo de retardo entre el bombeo y la prueba}\\
\'Esta tecnica ha sido utilizada con exito en el estudio de microemulsiones\cite{andy} (soluciones de una sola fase, \'opticamente transparentes y termodin\'amicamente estables) debido a las crecientes aplicaci\'ones comerciales de \'estas como son\cite{microemul}:
\begin{itemize}
\item Recuperaci\'on de petroleo
\item Recubrimiento de textiles
\item Detergentes
\item Industria Alimentaria
\item Cosm\'eticos
\item Confinamiento de elementos org\'anicos
\end{itemize}

Ya que la di\'amica molecular en l\'iquidos es un proceso utrarr\'apido (con una duración temporal de $10^{-15}$ segundos) es preciso utilizar un l\'aser de pulsos ultracortos. La cavidad l\'aser de un Titanio Safiro as\'i como la generac\'ion de pulsos ultracortos esta descrita en el cap\'itulo 2.\\

La descripci\'on del la t\'ecnica de detecci\'on por efecto Kerr \'optico es descrito en la secci\'on ??? del cap\'itulo 2. El montaje experimental tradicional es discutido asi como un nuevo montaje experimental para incrementar la sensibilidad es descrito en la secci\'on ???.\\

La instrumentaci\'on del montaje es discutida en la secci\'on ?? donde se describe la metodología para encontrar el restardo entre los pulsos de prueba y bombeo por medio del empleo de un absorbedor saturable.\\

As\'i mismo se muestra la te\'orica involucrada en el la medici\'on del ancho del pulso utilizado, las limitaciones que tenemos al utilizar un pulso del ancho encontrado.\\

\textcolor{blue}{En el cap\'itulo 2 se muestran los datos obtenidos y son analizados por medios de....}

\textcolor{blue}{En el cap\'itulo 2 se concluye a partir de los resultados obtenidos y se expone el trabajo a futuro.}


%\let\clearpage\relax
\begin{thebibliography}{2}
\bibitem{steve} \textsc{Neil A. Smith} y \textsc{Stephen R. Meech}, \textit{Optically-heterodyne-detected optical Kerr effect (OHD-OKE)}, Int. Reviews in Physical Chemistry, Vol 21, No. 1, 75-100 (2002).
\bibitem{andy} \textsc{Andrew A. Jaye}, \textit{Ultrafast Dynamics in the Dispersed Phase of oil-in-water microemulsions}, Thesis PHD, Univ. of East Anglia, School of Chemical Sciences and Pharmacy (2004)

\bibitem{microemul} \textsc{Paul Bidyut K.} y \textsc{Moulik Satya P.},
\textit{Uses and applications of microemulsions}, Current Science. {\bf 80}, 990–1001 (2001).
\end{thebibliography}
