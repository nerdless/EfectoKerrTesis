\documentclass[11pt,spanish]{report}
%\usepackage[square]{natbib}
\usepackage[spanish,mexico]{babel}
\usepackage{UNAMThesisIng}
\usepackage{amsmath}
\usepackage{amsfonts}
\usepackage{amssymb}
\usepackage[colorlinks,linkcolor=black,citecolor=black,urlcolor=blue, filecolor=urlcolor]{hyperref}
\usepackage[usenames,dvipsnames]{pstricks}
\usepackage{pst-circ}
\usepackage{epsfig}
\usepackage{pst-grad} % For gradients
\usepackage{pst-plot} % For axes
\usepackage{color}
\usepackage[all]{xy}
\usepackage{pst-optexp}
\usepackage{longtable, fancyhdr}

\logounam{Escudo-UNAM}
\logoinstitute{Escudo-IBT}
\pagenumbering{roman}
\flushbottom
\newtheorem{theorem}{Theorem}
\newtheorem{acknowledgement}[theorem]{Acknowledgement}
\newtheorem{algorithm}[theorem]{Algorithm}
\newtheorem{axiom}[theorem]{Axiom}
\newtheorem{case}[theorem]{Case}
\newtheorem{claim}[theorem]{Claim}
\newtheorem{conclusion}[theorem]{Conclusion}
\newtheorem{condition}[theorem]{Condition}
\newtheorem{conjecture}[theorem]{Conjecture}
\newtheorem{corollary}[theorem]{Corollary}
\newtheorem{criterion}[theorem]{Criterion}
\newtheorem{definition}[theorem]{Definition}
\newtheorem{example}[theorem]{Example}
\newtheorem{exercise}[theorem]{Exercise}
\newtheorem{lemma}[theorem]{Lemma}
\newtheorem{notation}[theorem]{Notation}
\newtheorem{problem}[theorem]{Problem}
\newtheorem{proposition}[theorem]{Proposition}
\newtheorem{remark}[theorem]{Remark}
\newtheorem{solution}[theorem]{Solution}
\newtheorem{summary}[theorem]{Summary}
\newenvironment{proof}[1][Proof]{\textbf{#1.} }{\ \rule{0.5em}{0.5em}}

\newcommand{\Cv}{\ensuremath{C_{V}}}
\newcommand{\gc}{^{\circ}\text{C}}
\newcommand{\cm}{\text{ cm}}
\newcommand{\invg}{\frac{1}{^{\circ}}}
\newcommand{\D}{\Delta}
\renewcommand{\d}{\delta}
\newcommand{\parc}[2]{\ensuremath{\frac{\partial #1}{\partial #2}}}
\newcommand{\m}{\text{ m}}
\newcommand{\atm}{\text{ atm}}
\newcommand{\K}{\text{K}}
\newcommand{\lt}{\text{lt}}
\newcommand{\mol}{\text{ mol}}
\newcommand{\gr}{\text{ gr}}
\newcommand{\kg}{\text{ kg}}
\newcommand{\pua}{\text{ \underline{P}}}
\newcommand{\nt}{\text{Nt}}
\newcommand{\sen}{\text{Sen}}
\renewcommand{\H}{\text{H}}
\renewcommand{\cos}{\text{Cos }}
\newcommand{\cte}{\text{cte}}
\newcommand{\J}{\text{ J}}

\hypersetup{pdftitle={Implementaci\'on de la t\'ecnica de detecci\'on Heterodina \'Optica por Efecto Kerr \'Optico (OHD-OKE) para el estudio de din\'amica ultrarr\'apida en l\'iquidos}, pdfauthor={Jos\'e Eduardo Ochoa Morales}, pdfsubject={F\'isico-Qu\'imica}, pdfkeywords={Espectrocop\'ia Ultrar\'apida, Femtosegundos, l\'iquidos moleculares}, pdfcreator={TeXnicCenter}, pdfproducer={MikteX}, baseurl={}}

\begin{document}

\title{Implementaci\'on de la t\'ecnica de detecci\'on Heterodina \'Optica por Efecto Kerr \'Optico (OHD-OKE) para el estudio de din\'amica ultrarr\'apida en l\'iquidos}
\author{Jos\'e Eduardo Ochoa Morales}
\prevdegrees{F\'isico, Universidad Nacional Aut\'onoma de M\'exico (2012)}
\institute{Programa de Maestr\'ia y Doctorado en Ingenier\'ia}
\department{Ingenier\'ia El\'ectrica - Instrumentaci\'on Cient\'ifica}
\degree{Maestro en Ingeniería}
\supervisor{Dr. Jes\'us Gardu\~no Mej\'ia}
\supervisorinst{CCADET-UNAM}
\city{M\'exico, Distrito Federal}
\degreemonth{Mes de Titulación}
\degreeyear{2014}
%Jurado
\presidente{presidente}
\secretario{secre}
\vocal{vocal}
\suplenuno{sup1}
\suplendos{sup2}
\lugar{Ciudad Universitaria, UNAM}


\maketitle

\begin{dedication}
A pesar de la distancia y del\\
tiempo ido, s\'{o}lo puedo dedicar:\\
A todos y ninguno...\\
\textsc{Julio A. Freyre-Gonz\'{a}lez}
\end{dedication}

\begin{acknowledgements}
Gracias a cada una de las personas que me apoyaron e hicieron que
este sue\~{n}o se cristalizara en una hermosa realidad.
\end{acknowledgements}

\tableofcontents
\clearpage



%\begin{foreword}
%Redacte aqu\'{\i} el pr\'{o}logo contando la historia detr\'{a}s de su
%trabajo.
%\end{foreword}

\begin{resumen}
Aqu\'{\i} se redacta el resumen en espa\~{n}ol.
\end{resumen}

\begin{abstract}
Here goes the english abstract.
\end{abstract}

\pagenumbering{arabic}

%\part{The First Part}

\chapter{Introducci\'on}

La Detecci\'on \'Optica por Efecto Kerr \'Optico (OKE)\cite{steve} es una t\'ecnica relativamente simple, no destructiva y no invasiva con excelente relaci\'on se\~nal-ruido y alta resoluci\'on temporal aplicada en muestras en fase l\'iquida para el estudio de la din\'amica molecular. El propósito de esta t\'esis es la implementación de dicha técnica para el estudio de la din\'amica molecular de CS$_2$.\\
\textcolor{red}{Aqu\'i no se si hablar de manera resumida de la t\'ecnica decir algo como}\\
\textcolor{blue}{En \'esta un pulso \'optico linealmente polarizado interacciona con la muestra que presenta un estado isotr\'opico inicial para posteriormente despu\'es de la perturbaci\'on induce una birrefringencia transitoria. Un segundo pulso, retrasado en tiempo incidiendo en el mismo punto en la muestra, con su plano de polarizaci\'on orientado a cierto \'angulo respecto al primero, experimentar\'a un cambio de polarizaci\'on debido a la birrefringencia inducida en el medio y emerger\'a con cierto grado de polarizaci\'on el\'iptica. Con un polarizador cruzado con el plano de polarizaci\'on del segundo pulso es colocado frente a un detector. La birrefringencia transitoria inducida en la muestra provocar\'a que cierta fracci\'on de la prueba llegue el detector. Esto es lo que se conoce como se\~nal OKE resuelta en tiempo ya que la relajaci\'on de la birrefringencia es monitoreada en del tiempo de retardo entre el bombeo y la prueba}\\
\'Esta tecnica ha sido utilizada con exito en el estudio de microemulsiones\cite{andy} (soluciones de una sola fase, \'opticamente transparentes y termodin\'amicamente estables) debido a las crecientes aplicaci\'ones comerciales de \'estas como son\cite{microemul}:
\begin{itemize}
\item Recuperaci\'on de petroleo
\item Recubrimiento de textiles
\item Detergentes
\item Industria Alimentaria
\item Cosm\'eticos
\item Confinamiento de elementos org\'anicos
\end{itemize}

Ya que la di\'amica molecular en l\'iquidos es un proceso utrarr\'apido (con una duración temporal de $10^{-15}$ segundos) es preciso utilizar un l\'aser de pulsos ultracortos. La cavidad l\'aser de un Titanio Safiro as\'i como la generac\'ion de pulsos ultracortos esta descrita en el cap\'itulo 2.\\

La descripci\'on del la t\'ecnica de detecci\'on por efecto Kerr \'optico es descrito en la secci\'on ??? del cap\'itulo 2. El montaje experimental tradicional es discutido asi como un nuevo montaje experimental para incrementar la sensibilidad es descrito en la secci\'on ???.\\

La instrumentaci\'on del montaje es discutida en la secci\'on ?? donde se describe la metodología para encontrar el restardo entre los pulsos de prueba y bombeo por medio del empleo de un absorbedor saturable.\\

As\'i mismo se muestra la te\'orica involucrada en el la medici\'on del ancho del pulso utilizado, las limitaciones que tenemos al utilizar un pulso del ancho encontrado.\\

\textcolor{blue}{En el cap\'itulo 2 se muestran los datos obtenidos y son analizados por medios de....}

\textcolor{blue}{En el cap\'itulo 2 se concluye a partir de los resultados obtenidos y se expone el trabajo a futuro.}


%\let\clearpage\relax
\begin{thebibliography}{2}
\bibitem{steve} \textsc{Neil A. Smith} y \textsc{Stephen R. Meech}, \textit{Optically-heterodyne-detected optical Kerr effect (OHD-OKE)}, Int. Reviews in Physical Chemistry, Vol 21, No. 1, 75-100 (2002).
\bibitem{andy} \textsc{Andrew A. Jaye}, \textit{Ultrafast Dynamics in the Dispersed Phase of oil-in-water microemulsions}, Thesis PHD, Univ. of East Anglia, School of Chemical Sciences and Pharmacy (2004)

\bibitem{microemul} \textsc{Paul Bidyut K.} y \textsc{Moulik Satya P.},
\textit{Uses and applications of microemulsions}, Current Science. {\bf 80}, 990–1001 (2001).
\end{thebibliography}


\chapter{Teor\'ia y M\'etodo experimental}
\textcolor{blue}{El objetivo de esta t\'esis es implementar la t\'ecnica de detecci�n \'Optica del Efecto Kerr \'Optico para el estudio de la din\'amica molecular en CS$_2$ }

\textcolor{red}{Poner una resumen del cap\'itulo}\\
En este cap\'itulo se expone la te\'oria y el montaje experimental del dispositivo. El cap\'itulo inicia con una descripcci\'on del sistema l\'aser y la manera en que \'este genera pulsos ultracortos.  Posteriormente se expone la teor\'ia de la detecci\'on Homodina por efecto Kerr \'Optico donde se muestra el arreglo experimental tradicional y la modificaci\'on hecha para aumentar la sensibilidad. Se expone la metodolog\'ia para encontrar el retardo cero y la te\'oria detr\'as del absorvedor saturable para encontrar el retardo. 

\section{Generaci\'on de Pulsos Ultracortos}
%exponer lo del principio de que necesito pulsos cortos para ver fen�menos cortos
%




%\begin{thebibliography}{X}
%\bibitem{Dan} \textsc{Dantzig, G.B.} y \textsc{P. Wolfe},
%<<Decomposition principle for linear programs>>,
%\textit{Operations Research}, \textbf{8}, p�ags. 101--111, 1960.
%\end{thebibliography}


\appendix 

\chapter{Medida del ancho de espot y cintura}
\section{Medida del spot de bombeo y prueba y su separaci\'on}
Para realizar la medida del tama\~no del espot utilic\'e el m\'etodo de la navaja \cite{rober} el cual consiste en bloquear el haz con una navaja en direcci\'on transversal a la propagaci\'on de la luz y obtener la intensidad despu\'es de la navaja como funci\'on de la distancia de penetraci\'on de la navaja 


\scalebox{1} % Change this value to rescale the drawing.
{
\begin{pspicture}(0,-2.14)(3.08,2.14)
\psbezier[linewidth=0.04](1.0,2.0)(1.0,1.25)(2.0,1.25)(2.0,2.0)
\psline[linewidth=0.04cm](0.0,2.0)(1.0,2.0)
\psline[linewidth=0.04cm](2.0,2.0)(3.0,2.0)
\psline[linewidth=0.04cm](3.0,2.0)(3.0,-2.0)
\psline[linewidth=0.04cm](3.0,-2.0)(2.0,-2.0)
\psbezier[linewidth=0.04](1.0,-2.0)(1.0,-1.25)(2.0,-1.25)(2.0,-2.0)
\psline[linewidth=0.04cm](0.0,2.0)(0.0,-2.0)
\psline[linewidth=0.04cm](0.0,-2.0)(1.0,-2.0)
\end{pspicture} 
}

\begin{pspicture}(5,4)
\psset{fillstyle=gradient,linestyle=none}
\pscircle[GradientCircle=true](5,1){1}%
\psframe[GradientCircle=true,GradientScale=3](0,1.5)(5,2.5)%
\psframe[GradientCircle=true,GradientScale=2,GradientPos={(4,3.5)}](0,3)(5,4)%
\end{pspicture}

\begin{pspicture}(6,4)
\psgrid
\psset{fillstyle=gradient,linestyle=none, gradend=red, gradbegin=white}
\psframe[gradmidpoint=0](0,1.5)(6,2.5)
\psframe[gradmidpoint=0.5](3,0)(6,1)
\psframe[gradmidpoint=1](0,0.5)(3,1)
\end{pspicture}

\begin{pspicture}(0,-1)(4,2)
\psgrid
\pnodes(0,1.5){A}(2,1.5){B}(2.8,0){C}
\mirror[mirrorwidth=1.2, mirrortype=extended](A)(B)(C)
\addtopsstyle{Beam}{%
fillstyle=solid, fillcolor=green, opacity=0.3}
\drawwidebeam[beamwidth=0.5](A){}
\drawwidebeam[loadbeam]{}(C)
\end{pspicture}

%\let\clearpage\relax
\begin{thebibliography}{2}
\bibitem{rober} \textsc{R. D\'iaz-Uribe, M. Rosete-Aguilar} y \textsc{R. Ortega-Mart\'inez},
\textit{Position sensing of a Gaussian beam with a power meter and a knife edge}, Rev. Mex. F\'is. {\bf 39}, 484–492 (1992).
\end{thebibliography}


%\bibliographystyle{UNAMThesis}
%\bibliography{testBib}
\end{document}