\textcolor{blue}{El objetivo de esta t\'esis es implementar la t\'ecnica de detecci�n \'Optica del Efecto Kerr \'Optico para el estudio de la din\'amica molecular en CS$_2$ }

\textcolor{red}{Poner una resumen del cap\'itulo}\\
En este cap\'itulo se expone la te\'oria y el montaje experimental del dispositivo. El cap\'itulo inicia con una descripcci\'on del sistema l\'aser y la manera en que \'este genera pulsos ultracortos.  Posteriormente se expone la teor\'ia de la detecci\'on Homodina por efecto Kerr \'Optico donde se muestra el arreglo experimental tradicional y la modificaci\'on hecha para aumentar la sensibilidad. Se expone la metodolog\'ia para encontrar el retardo cero y la te\'oria detr\'as del absorvedor saturable para encontrar el retardo. 

\section{Generaci\'on de Pulsos Ultracortos}
%exponer lo del principio de que necesito pulsos cortos para ver fen�menos cortos
%




%\begin{thebibliography}{X}
%\bibitem{Dan} \textsc{Dantzig, G.B.} y \textsc{P. Wolfe},
%<<Decomposition principle for linear programs>>,
%\textit{Operations Research}, \textbf{8}, p�ags. 101--111, 1960.
%\end{thebibliography}
